\documentclass[11pt]{article}
\pagestyle{empty}
\usepackage{color}
\usepackage{fancyhdr}
\usepackage{lastpage}
\usepackage{amsfonts}
\usepackage{amsmath}
\usepackage{enumitem}
\pagestyle{fancy}
\renewcommand{\headrulewidth}{0pt}
\cfoot[R]{\thepage~of~\pageref{LastPage}}
\addtolength{\oddsidemargin}{-.875in}
\addtolength{\evensidemargin}{-.875in}
\addtolength{\textwidth}{1.75in}
\addtolength{\topmargin}{-.875in}
\addtolength{\textheight}{1.75in}

\begin{document}
\begin{center}
\textbf{Lambda calculus}
\end{center}
\\ $(\lambda \textcolor{red}{x}\ .\ \textcolor{red}{x}\ \textcolor{blue}{y})$ has \textcolor{red}{x bound} and \textcolor{blue}{y free}
\begin{itemize}
    \item x is \textcolor{blue}{FREE} in:
    \begin{itemize}
        \item $x$ if it is not bound in any other variable or constant
        \item $(E,F) \Leftrightarrow (x$ \textcolor{blue}{free} in E$)$ OR $(x$ \textcolor{blue}{free} in F$)$
        \item $(\lambda y\ .\ E) \Leftrightarrow 
            \left\{
                \begin{array}{ll}
                    x \mbox{ \textcolor{blue}{free} in E} \\
                    x \mbox{ and } y \mbox{ are different variables}
                \end{array}
            \right.$
    \end{itemize}
    \item x is \textcolor{red}{BOUND} in:
    \begin{itemize}
        \item $(E,F) \Leftrightarrow (x$ \textcolor{red}{bound} in E$)$ OR $(x$ \textcolor{red}{bound} in F$)$
        \item $(\lambda y\ .\ E) \Leftrightarrow (x$ \textcolor{red}{bound} in E$)$ \textbf{OR} 
            $\left\{
                \begin{array}{ll}
                    x \mbox{ \textcolor{blue}{free} in E} \\
                    x \mbox{ and } y \mbox{ are the same variable}
                \end{array}
            \right.$
    \end{itemize}
\end{itemize}
\\ \textbf{Substitutions:}
\begin{itemize}
    \item $x[M/x] = M$
    \item $y[M/x] = y$, where $y$ is a constant or variable other than $x$.
    \item $(E\ F)[M/x] = (E[M/x])(F[M/x])$
    \item $(\lambda x\ .\ E)[M/x] = \lambda x\ .\ E$
    \item $(\lambda y\ .\ E)[M/x] = 
            \left\{
                \begin{array}{ll}
                    \lambda y\ .\ (E[M/x]) if 
                                            \left\{
                                                \begin{array}{ll}
                                                    x \mbox{ not \textcolor{blue}{free} in E} \\
                                                    \ \ \ \ \ \ \mbox{OR} \\
                                                    y \mbox{ not \textcolor{blue}{free} in M}
                                                \end{array}
                                            \right. \\
                    \ \ \ \ \ \ \mbox{OR} \\
                    \lambda z\ .\ (E[z/y])[M/x] \mbox{ otherwise, where z is new and not \textcolor{blue}{free} in E or M}
                \end{array}
            \right.$
\end{itemize}
\\ \textbf{Conversions:}
\begin{itemize}
    \item \textbf{$\alpha$-conversion:} If $y$ not \textcolor{blue}{free} in E, $(\lambda x\ .\ E) \Leftrightarrow (\lambda y\ .\ E[y/x])$
    \item \textbf{$\beta$-conversion:} $(\lambda x\ .\ E)M \Leftrightarrow E[M/x])$
    \item \textbf{$\eta$-conversion:} If 
                                        $\left\{
                                            \begin{array}{ll}
                                                x \mbox{ not \textcolor{blue}{free} in E} \\
                                                E \mbox{ denotes a function}
                                            \end{array}
                                        \right.$, then $(\lambda x\ .\ Ex) \Leftrightarrow E$
\end{itemize}
\\ \textbf{Reductions:}
$\beta$-conversion and $\eta$-conversion are called \textbf{reductions} when used in normal order.
\\ In example, $(\lambda x\ .\ E)M$ can be $\beta$-reduced and is therefore called a reducible expression, or \textbf{redex}.
\\ When an expression is no longer a redex, it is in its \textbf{normal form}.
\\ Example of an expression which reduces to normal form under normal-order evaluation, but not under applicative-order evaluation:
\\ The expression $(\lambda y\ .\ 1)((\lambda x\ .(x\ x))(\lambda x\ .(x\ x)))$.
\begin{itemize}
    \item Using \textbf{normal-order evaluation}:
    \\ $(\lambda y\ .\ 1)((\lambda x\ .(x\ x))(\lambda x\ .(x\ x))) \rightarrow y$
    \item Using \textbf{applicative-order evaluation}:
    \\ $(\lambda y\ .\ 1)((\lambda x\ .(x\ x))(\lambda x\ .(x\ x))) \rightarrow (\lambda y\ .\ 1)((\lambda x\ .(x\ x))(\lambda x\ .(x\ x)))$ ... (infinite) 
\end{itemize}
\\ \textbf{Y-combinator:}
\\ The Y combinator is represented by 
$Y = (\lambda h.(\lambda x . h(x\ x))(\lambda x.h(x\ x)))$
\\ $Y f = (\lambda h\ .\ (\lambda x\ .\ h(x\ x))(\lambda x\ .\ h(x\ x)))\ \textbf{f}$
\\ $ \Leftrightarrow_\beta (\lambda x\ .\ \textbf{f}(x\ x))(\lambda x\ .\ \textbf{f}(x\ x)))$
\\ $ \Leftrightarrow_\beta \textbf{f}\ (\lambda x\ .\ f(x\ x))(\lambda x\ .\ f(x\ x)))$
\\ Since $\Leftrightarrow_\beta$ is symmetric:
\\ $ \Leftrightarrow_\beta f\ (\textbf{Y f})$
\\ Therefore, $Y\ f \Leftrightarrow_\beta f\ (Y\ f)$
\\\\ Example of the definition of a recursive function using the Y-combinator and show how it is recursive with reductions.
\\ Let f = ($\lambda$f\ .\ $\lambda$x\ .\ if ($=$ x 0) 0 (if ($=$ x 1) 1 (+ (f (- x 1)) (f (- x 2)))))
\\ The Fibonacci function is then FIB = Y f = f (Y f) = f FIB (for all f)
\\ For example, FIB 3 = f (Y f) 3
\\ $\rightarrow$ f FIB 3
\\ $\rightarrow$ ($\lambda$x\ .\ if ($=$ x 0) 0 (if ($=$ x 1) 1 (+ (FIB (- x 1)) (FIB (- x 2))))) 3
\\ $\rightarrow$ (if ($=$ 3 0) 0 (if ($=$ 3 1) 1 (+ (FIB (- 3 1)) (FIB (- 3 2)))))
\\ $\rightarrow$ (if false 0 (if ($=$ 3 1) 1 (+ (FIB (- 3 1)) (FIB (- 3 2)))))
\\ $\rightarrow$ (if false 0 (if false 1 (+ (FIB (- 3 1)) (FIB (- 3 2)))))
\\ $\rightarrow$ (+ (FIB (- 3 1)) (FIB (- 3 2)))
\\ $\rightarrow$ (+ (FIB 2) (FIB (- 3 2)))
\\ $\rightarrow$ (+ (FIB 2) (FIB 1))
\\ $\rightarrow$ (+ (($\lambda$x\ .\ if ($=$ x 0) 0 (if ($=$ x 1) 1 (+ (FIB (- x 1)) (FIB (- x 2))))) 2) (($\lambda$x\ .\ if ($=$ x 0) 0 (if ($=$ x 1) 1 (+ (FIB (- x 1)) (FIB (- x 2))))) 1))
\\ $\rightarrow$ ...
\\ $\rightarrow$ (+ (+ (FIB 1) (FIB 0)) 1)
\\ $\rightarrow$ ...
\\ $\rightarrow$ 2
\\\\ \textbf{Church-Rosser theorems:}
\begin{itemize}
    \item [Th1] An expression cannot be reduced to two distinct normal forms.
    \item [Th2] If there exists a reduction sequence from an expression to a normal form, then the normal order reduction will also reduce it to normal form. (normal order reduction is most likely to \textbf{terminate}).
\end{itemize}



\end{document}
