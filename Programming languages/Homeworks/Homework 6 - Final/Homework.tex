\documentclass[10pt]{article}
\pagestyle{empty}
\usepackage{color}
\usepackage{fancyhdr}
\usepackage{lastpage}
\usepackage{minted}
\pagestyle{fancy}
\renewcommand{\headrulewidth}{0pt}
\cfoot[R]{\thepage~of~\pageref{LastPage}}
\addtolength{\oddsidemargin}{-.875in}
\addtolength{\evensidemargin}{-.875in}
\addtolength{\textwidth}{1.75in}
\addtolength{\topmargin}{-.875in}
\addtolength{\textheight}{1.75in}

\title{Programming Languages \\ CSCI-GA.2110.001 Spring 2017 \\\\ Homework 2 \\ Due Monday, May 8}
\author{Quentin McGaw (qm301)}
\date{May 8, 2017}

\begin{document}
\maketitle
\begin{enumerate}
    \item 
    \begin{enumerate}
        \item \textbf{\textcolor{blue}{In the $\lambda$-calculus, give an example of an expression which would reduce to normal form under normal-order evaluation, but not under applicative-order evaluation.}}
            \\ The expression $(\lambda y\ .\ 1)((\lambda x\ .(x\ x))(\lambda x\ .(x\ x)))$.
            \begin{itemize}
                \item Using \textbf{normal-order evaluation}:
                \\ $(\lambda y\ .\ 1)((\lambda x\ .(x\ x))(\lambda x\ .(x\ x))) \rightarrow 1$
                \item Using \textbf{applicative-order evaluation}:
                \\ $(\lambda y\ .\ 1)((\lambda x\ .(x\ x))(\lambda x\ .(x\ x))) \rightarrow (\lambda y\ .\ 1)((\lambda x\ .(x\ x))(\lambda x\ .(x\ x))) \rightarrow$ ... (infinite loop) 
            \end{itemize}
        \item \textbf{\textcolor{blue}{Write the definition of a recursive function (other than factorial) using the Y combinator. Show a series of reductions of an expression involving that function which illustrates how it is, in fact, recursive (as I did in class for factorial).}}
            \\ Let f = ($\lambda$f\ .\ $\lambda$x\ .\ if ($=$ x 0) 0 (if ($=$ x 1) 1 (+ (f (- x 1)) (f (- x 2)))))
            \\ The Fibonacci function is then FIB = Y f = f (Y f) = f FIB (for all f)
            \\ For example, FIB 3 = f (Y f) 3
            \\ $\rightarrow$ f FIB 3
            \\ $\rightarrow$ ($\lambda$x\ .\ if ($=$ x 0) 0 (if ($=$ x 1) 1 (+ (FIB (- x 1)) (FIB (- x 2))))) 3
            \\ $\rightarrow$ (if ($=$ 3 0) 0 (if ($=$ 3 1) 1 (+ (FIB (- 3 1)) (FIB (- 3 2)))))
            \\ $\rightarrow$ (if false 0 (if ($=$ 3 1) 1 (+ (FIB (- 3 1)) (FIB (- 3 2)))))
            \\ $\rightarrow$ (if false 0 (if false 1 (+ (FIB (- 3 1)) (FIB (- 3 2)))))
            \\ $\rightarrow$ (+ (FIB (- 3 1)) (FIB (- 3 2)))
            \\ $\rightarrow$ (+ (FIB 2) (FIB (- 3 2)))
            \\ $\rightarrow$ (+ (FIB 2) (FIB 1))
            \\ $\rightarrow$ (+ (($\lambda$x\ .\ if ($=$ x 0) 0 (if ($=$ x 1) 1 (+ (FIB (- x 1)) (FIB (- x 2))))) 2) (($\lambda$x\ .\ if ($=$ x 0) 0 (if ($=$ x 1) 1 (+ (FIB (- x 1)) (FIB (- x 2))))) 1))
            \\ $\rightarrow$ ...
            \\ $\rightarrow$ (+ (+ (FIB 1) (FIB 0)) 1)
            \\ $\rightarrow$ ...
            \\ $\rightarrow$ 2
        \item \textbf{\textcolor{blue}{Write the actual expression in the $\lambda$-calculus representing the Y combinator, and show that it satisfies the property Y(f) = f(Y(f)).}}
            \\ The Y combinator is represented by 
            $Y = (\lambda h\ .\ (\lambda x\ .\ h(x\ x))(\lambda x\ .\ h(x\ x)))$ in the lambda calculus.
            \\ $Y f = (\lambda h\ .\ (\lambda x\ .\ h(x\ x))(\lambda x\ .\ h(x\ x)))\ \textbf{f}$
            \\ $ \Leftrightarrow_\beta (\lambda x\ .\ \textbf{f}(x\ x))(\lambda x\ .\ \textbf{f}(x\ x)))$
            \\ $ \Leftrightarrow_\beta \textbf{f}\ (\lambda x\ .\ f(x\ x))(\lambda x\ .\ f(x\ x)))$
            \\ Since $\Leftrightarrow_\beta$ is symmetric:
            \\ $ \Leftrightarrow_\beta f\ (\textbf{Y f})$
            \\ Therefore, $Y\ f \Leftrightarrow_\beta f\ (Y\ f)$
        \item \textbf{\textcolor{blue}{Summarize, in your own words, what the two Church-Rosser theorems state.}}
            \begin{itemize}
                \item [Theorem 1] An expression cannot be reduced to two distinct normal forms.
                \item [Theorem 2] If there exists a reduction sequence from an expression to a normal form, then the normal order reduction will also reduce it to normal form. (normal order reduction is most likely to \textbf{terminate}).
            \end{itemize}
    \end{enumerate}
    
    
    \item
    \begin{enumerate}
        \item \textbf{\textcolor{blue}{In ML, why do all lists have to be homogeneous (i.e. all elements of a list must be of the same type)?}}
            \\ Because ML uses \textbf{static type checking}, the compiler must be able to infer the type of each value at compile time. 
            \\ If lists were heterogeneous, the compiler could not infer the type of an element selected from a list.
        \item \textbf{\textcolor{blue}{Write a function in ML whose type is ('a $->$ 'b list) $->$ ('b $->$ 'c list) $->$ 'a $->$ 'c.}}
            \begin{minted}{sml}
fun custom f g h x = 
    let val (y::ys) = g x
        val (z::zs) = h y
    in z
    end;
            \end{minted}
        \item \textbf{\textcolor{blue}{What is the type of the following function (try to answer without running the ML
        system)?}}
        \begin{minted}{sml}
fun foo (op >) x (y,z) =
    let fun bar a = if x > y then z else a
    in bar [1,2,3]
    end
        \end{minted}
            \\ See the next question for explanations
            \\ The type of foo is : fn : ('a * 'b $->$ bool) $->$ 'a $->$ 'b * int list $->$ int list
        \item \textbf{\textcolor{blue}{Provide an intuitive explanation of how the ML type inferencer would infer the type that you gave as the answer to the previous question.}}
            \begin{itemize}
                \item The function bar defined in foo compares x and y with the infix $>$. No further constraint on x or y are given so the inferencer assigns them the types 'a and 'b respectively. The output of this infix is used as a boolean and is therefore assigned the type bool. The type of the infix is hence ('a * b' $->$ bool).
                \item The function bar defined in foo takes the argument a and returns z or a. In foo, it only gets the argument [1,2,3] which is an int list. Therefore bar must return an int list so z is of type int list.
                \item Finally, because foo returns what bar returns, it returns a type int list.
                \item This gives foo's type: fn : ('a * 'b $->$ bool) $->$ 'a $->$ 'b * int list $->$ int list
            \end{itemize}      
    \end{enumerate}
    
    
    \item
    \begin{enumerate}
        \item \textbf{\textcolor{blue}{As discussed in class, what are the three features that a language must have in order to be considered object oriented?}}
            \begin{itemize}
                \item \textbf{Encapsulation of data and code (methods):} Creation of self-contained modules binding processing functions to the data (classes, object). It keeps routines separate and less prone to conflict with each other.
                \item \textbf{Inheritance:} Allows to re-use the structure, attributes and methods of an existing class into another class down the hierarchy (child).
                \item \textbf{Subtyping with dynamic dispatch} (select which implementation of a polymorphic method should be called at run time) 
            \end{itemize}
        \item \textbf{\textcolor{blue}{What is the "subset interpretation of subtyping"?}}
            \\ The set of values defined by a subtype of a parent type is a subset of the set of values defined by the parent type. 
            \\ In other words if B is a subtype of A, then any value in the set of values defined by type B is also in the set of values defined by type A.
            \\ In a shorter way, $B <: A \Rightarrow S_B \subset S_A$
        \item \textbf{\textcolor{blue}{Explain why function subtyping must be contravariant in the parameter type and co-variant in the result type. If necessary, provide examples to illuminate your explanation.}}
            \begin{itemize}
                \item \textbf{Co-variant} preserves the A is a subtype of B hierarchy for the type of the results parameter.
                \item \textbf{Contravariant} reverses the type hierarchy for the arguments parameter.
            \end{itemize}
            \begin{itemize}
                \item Scala example to show covariant subtyping on the input would lead to an error.
                \begin{minted}{scala}
class A { val x = 2 }
class B extends A { val y = 3 }
var f: A=>Int
def g(b:B):Int = b.y; // type B=>Int
                \end{minted}
                If B$=>$Int is a subtype of A$=>$Int, we can do
                \begin{minted}{scala}
f = g
f(new A())
                \end{minted}
                because f is of type A$=>$Int but g would fail to access the field y of the A object.
                \item Scala example to show contravariant subtyping on the input type of a function is safe.
                \begin{minted}{scala}
class A { val x = 2 }
class B extends A { val y = 3 }
var f: B=>Int
def g(a:A):Int = a.y; // type B=>Int
                \end{minted}
                If A$=>$Int is a subtype of B=>Int, we can do
                \begin{minted}{scala}
f = g
f(new B())
                \end{minted}
                because f is of type B$=>$Int and g would work (B object is an A object)
                \item Scala example to show covariant on the output type of a function is safe, while contravariant isn't.
                \begin{minted}{scala}
class A { val x = 2 }
class B extends A { val y = 3 }
var fA: Int=>A
var fB: Int=>B
def gA(x:Int) = new A()
def gB(x:Int) = new B()

fA = gB // Covariance: Int=>B written to Int=>A
val a:A = fA(1); // returns B (from gB) which works as we can write a B object to an A variable

fB = gA // Contravariance: Int=>A written to Int=>B
val b:B = fB(1); // returns A (from gA) but can't write an A object to a B variable
def g(b:B):Int = b.y; // type B=>Int
                \end{minted}
            \end{itemize}
            
            
        \item \textbf{\textcolor{blue}{Provide an intuitive answer showing why function subtyping satisfies the subset interpretation of subtyping. Be sure to consider both the contravariant and covariant aspects of function subtyping.}}
            \begin{itemize}
                \item \textbf{Contravariance on parameter type}: 
                \\ If $B$ is a subtype of $A$, the type $A \rightarrow T$ denotes the set of all functions that can be applied to an A object and return a T. The type $B \rightarrow T$ denotes the set of all functions that can be applied to a B object and returns an int. Any function that can be applied to an A object is in the set denoted by $A \rightarrow T$ and can be applied to a B object and is also in $B \rightarrow T$ as a consequence. Since every element of $A \rightarrow T$ is also in $B \rightarrow T$, $A \rightarrow T \subset B \rightarrow T$.
                \item \textbf{Covariance on result type}: 
                \\ If $B$ is a subtype of $A$, any function in $T \rightarrow B$ returns a B object. Since a B object is an A object, it also returns an A object. Since every element of $T \rightarrow B$ is also in $T \rightarrow A$, $T \rightarrow B \subset T \rightarrow A$.
            \end{itemize}
        \item \textbf{\textcolor{blue}{Give an example in Scala that demonstrates subtyping of functions, utilizing both the contravariance on the parameter type and covariance on the result type.}}
        \\ The following Scala code has a first part showing the contravariance on the parameter type, and a second part demonstrating the covariance on the result type. \\
            \begin{minted}{scala}
class A
class B extends A
object Answer{
    def fAB(a:A):B = new B()
    def fBA(b:B):A = new A()
    var ba:B=>A = fAB // works, A->B subtype of B->A
    var ab:A=>B = fba // error, B->A not a subtype of A->B
}
            \end{minted}        
    \end{enumerate}
    
    
    \item \textbf{\textcolor{blue}{In Java generics, subtyping on instances of generic classes is invariant. That is, two different instances $C<A>$ and $C<B>$ of a generic class C have no subtyping relationship, regardless of a
    subtyping relationship between A and B (unless, of course, A and B are the same class).}}
    \begin{enumerate}
        \item \textbf{\textcolor{blue}{Write a function (method) in Java that illustrates why, even if B is a subtype of A, $C<B>$ should not be a subtype of $C<A>$. That is, write some Java code that, if the compiler allowed such covariant subtyping among instances of a generic class, would result in a run-time type error.}}
            \begin{minted}{java}
class A{};
class B extends A{};

void f(ArrayList<A> L){
    A a = new A;
    L.add(a); // if L is an ArrayList<B>, this would create an error later when A is thought as B.
}
            \end{minted}
        \item \textbf{\textcolor{blue}{Modify the code you wrote for the above question that illustrates how Java allows a form of polymorphism among instances of generic classes, without allowing subtyping. That is, make the function you wrote above be able to be called with many different instances of a generic class.}}
            \begin{minted}{java}
class A{};
class B extends A{};

void f(ArrayList<? super B> L){
    B b = new B;
    L.add(b); // if L is an ArrayList<A> for example, this would work.
}
            \end{minted}
    \end{enumerate}
    
    
    \item
    \begin{enumerate}
        \item \textbf{\textcolor{blue}{In Scala, write a generic class definition that supports covariant subtyping among instances of the class. For example, define a generic class C[E] such that if class B is a subtype of class A, then C[B] is a subtype of C[A].}}
            \begin{minted}{scala}
class C[+E]
            \end{minted}
            (Be \textbf{careful}, E should only be used to output of methods (covariant), not to inputs or assignments of \textbf{methods only}).
        \item \textbf{\textcolor{blue}{Give an example of the use of your generic class.}}
            \begin{minted}{scala}
// To test it:
class A
class B extends A
val c: C[A] = new C[B]() // This works
val c: C[B] = new C[A]() // This gives a type error
            \end{minted}
        \item \textbf{\textcolor{blue}{In Scala, write a generic class definition that supports contravariant subtyping among instances of the class. For example, define a generic class C[E] such that if class B is a subtype of class A, then C[A] is a subtype of C[B].}}
            \begin{minted}{scala}
class C[-E]
            \end{minted}
            (Be \textbf{careful}, E should only be used at input types of methods (contravariant))
        \item \textbf{\textcolor{blue}{Give an example of the use of your generic class.}}
            \begin{minted}{scala}
// To test it:
class A
class B extends A
val c: C[A] = new C[B]() // This gives a type error
val c: C[B] = new C[A]() // This works
            \end{minted}
        \item \textbf{\textcolor{blue}{Consider the following Scala definition of a tree type, where each node contains a value.}}
        \begin{minted}{scala}
abstract class Tree[T <: Ordered[T]]
case class Node[T <: Ordered[T]](v:T, l:Tree[T], r:Tree[T]) extends Tree[T]
case class Leaf[T <: Ordered[T]](v:T) extends Tree[T]
        \end{minted}
        \textbf{\textcolor{blue}{Ordered is a built-in trait in Scala (see \\ http://www.scala-lang.org/api/current/scala/math/Ordered.html). Write a Scala function that takes a Tree[T], for any ordered T, and returns the smallest (minimum) value in the tree. Be sure to use good Scala programming style.}}
        \begin{minted}{scala}
def minimum[T <: Ordered[T]](tr:Tree[T]):T = {
  tr match {
    case Leaf(v) => v
    case Node(v, l, r) => {
      var min_left = minimum(l)
      var min_right = minimum(r)
      if(min_left < min_right){
        return min_left
      }else{
        return min_right
      }
    }
  }
}
        \end{minted}
    \end{enumerate}
    
    
    \item
    \begin{enumerate}
        \item \textbf{\textcolor{blue}{What is the advantage of a reference counting collector over a mark and sweep collector?}}
            \begin{itemize}
                \item \textbf{Reference counting collector: } Determine during execution when a particular object has become dead. At that point, the storage for that object is reclaimed.
                \item \textbf{Mark and sweep collector: } When the heap is full, it traverses memory to determine the set of all live objects. The complement of this set can be reclaimed (dead objects).
            \end{itemize}
            For a reference counting collector:
            \begin{itemize}
                \item Its cost is O($\#$ pointers created) + O($\#$ pointers destroyed) which is spread over the whole execution which might be lower than the cost of the mark and sweep collector.
                \item Storage reclamation is incremental, and not just when the heap is full as for the mark and sweep collector.
                \item There is no long program suspension (happens a little bit at a time), whereas the mark and sweep collector "stop-and-collect" (not suitable for real-time applications).
            \end{itemize}
        \item \textbf{\textcolor{blue}{What is the advantage of a copying garbage collector over a mark and sweep garbage collector?}}
            \\ \textbf{Copying garbage collector: } There are 2 heaps named FROM and TO, only one used at a time. When the FROM space is full, the garbage collection begins through the graph and each live object is copied to the TO space (using compaction with a heap pointer to allocate the heap). FROM and TO are then flipped around and program execution resumes.
            For a copying garbage collector:
            \begin{itemize}
                \item The cost is O(L+M)=O(L), where L is the amount of live storage and M the size of the heap which is usually three times bigger than L. In contrast, the cost of the mark and sweep GC is O(L) + O(M) = O(M), which is proportional to the size of the heap M and worst.
            \end{itemize}
        \item \textbf{\textcolor{blue}{Write a brief description of generational copying garbage collection.}}
            \\ \textbf{Generational copying garbage collection:} It uses a serie of heaps, called generations, where each generation contains objects of a similar "age" (time they lived so far). When a generation fills up, its live objects are copied to the next generation. This was designed to make profit of the discovery that  "young" objects tend to die in a short time while "old" objects tend to live longer. As a consequence, this type of GC runs mostly in the youngest generation and will rarely process the older generations, hence making it happen less frequently and more efficiently.
        \item \textbf{\textcolor{blue}{Write, in the language of your choice, the procedure delete(x) in a reference counting GC system, where x is a pointer to a structure (e.g. object, struct, etc.) and delete(x) deletes the pointer x. Assume that there is a free list of available blocks and addToFreeList(x) puts the structure that x points to onto the free list.}}
        \\ The following code is written in C++ :
        \begin{minted}{c++}
void delete(Object* x){
    x->refcount = x->refcount - 1;
    if (x->refcount == 0){
        for(int i = 0; i < sizeof(x->pointers)/sizeof(x->pointers[0]); i++){
            delete(x->pointers[i]);
        }
        addToFreeList(x);
    }
}
        \end{minted}
    \end{enumerate}
\end{enumerate}

\end{document}
