\documentclass[10pt]{article}
\pagestyle{empty}
\usepackage{minted}
\usepackage{fancyhdr}
\usepackage{lastpage}
\pagestyle{fancy}
\renewcommand{\headrulewidth}{0pt}
\cfoot[R]{\thepage~of~\pageref{LastPage}}
\addtolength{\oddsidemargin}{-.875in}
\addtolength{\evensidemargin}{-.875in}
\addtolength{\textwidth}{1.75in}
\addtolength{\topmargin}{-.875in}
\addtolength{\textheight}{1.75in}

\begin{document}

\begin{minted}[fontsize=\normalsize]{java}
import java.util.*;

/* 1. Define a generic interface Addable<>, parameterized by a type T. The Addable<> 
      interface should only require a method plus, which takes as a parameter an 
      object of type T and returns an object of type T. */
interface Addable<T> {
    public T plus(T o);
}

/* 2. Define a generic class MyList<> that derives from the built�in ArrayList<> class 
and implements both the Comparable<> and Addable<> interfaces, such that two objects of 
type MyList<T>, for a suitable type T, can be compared using the compareTo method and 
can be added using the plus method (see above). Furthermore, the definition of MyList<> 
must require that its type parameter T itself must implement the Comparable<> and 
Addable<> interfaces, so that two T objects can be compared and added.

Within MyList<>, the comparison method, compareTo(), required by the Comparable interface, 
should take another MyList<> object (for the same type parameter T) and perform a comparison 
based on the sum (using the plus method of the Addable<> interface) of the elements in the 
two MyList<> objects. That is, a MyList<T> object L1 is less than an MyList<T> object L2, 
for the same T, if the sum of the elements in L1 is less than the sum of the elements of L2. 
Alternatively, L1 is equal to L2 if the sums of the elements of the two lists are equal. 
Otherwise L1 is greater than L2.

The plus operation for MyList should implement concatentation. That is, the result of 
L1.plus(L2) returns a new MyList containing the elements of L1 followed by the elements of L2. 
Neither L1 nor L2 should be modified.

The constructor for MyList<T> should be of the form MyList(T z) { ... } where z represents the 
"zero" (additive identity) value of type T. This is the value that should be used as the sum 
of the list of elements in a MyList<T> (such as for comparison purposes) when the list is empty.

You may override the toString() method, if you don't like the way your MyList objects print. */
class MyList<T extends Addable<T> & Comparable<T>>
		extends ArrayList<T>
		implements Comparable<MyList<T>>, Addable<MyList<T>>{
	T Z;		
	public int compareTo(MyList<T> L2) {
		T sum2 = Z;
		for (T e : L2){
			sum2 = sum2.plus(e);
		}
		T sum1 = Z;
		for (T e : this){
			sum1 = sum1.plus(e);
		}
		return sum1.compareTo(sum2);
	}
	public MyList<T> plus(MyList<T> L2) {
		MyList<T> newlist = new MyList<T>(Z);
		newlist.addAll(this);
		newlist.addAll(L2);			
		return newlist;
	}
	MyList(T z){
		Z = z;
	}
	public String toString() {
        String s = "";
		for (T e : this){
            s += " "+e.toString()+" ";
		}
		return "[" + s + "]";  
	}
}

/* 3. Define a class A that can be used to instantiate MyList<A>, which also means that two 
A's must be able to be compared to each other and added together. You can define A any way 
you like, the only requirements are: 
- A includes a constructor, A(Integer x) {...}.
- when comparing two A objects, the result of the comparison should be based on comparing 
  the x values that the two objects were initially constructed with. That is, given
  A a1 = new A(6);
  A a2 = new A(7);
  the result of a1.compareTo(a2) should return �1, indicating that a1 is less than a2. 
  Furthermore, the result of a1.plus(a2) should be a new A object whose constructor is passed 
  13 (i.e. 6+7). You'll also want to override the toString() method, so A objects print nicely.
*/
class A implements Comparable<A>, Addable<A>{
	Integer X;
	A(Integer x){
		X = x;}
	public int compareTo(A o){
		if (this.X < o.X){
			return -1;
		}else if(this.X > o.X){
			return 1;
		}else{
			return 0;
		}
	}
	public A plus(A o){
		return new A(this.X + o.X);
	}
	public String toString() { 
		return "A<" + X + ">";  
	}
}

/* 4. Define a class B that derives from class A. Other than overriding the toString() method 
and providing a constructor that takes an integer, you don't need to add anything to class B. */
class B extends A{
	B(Integer x){
		super(x);
	}
	public String toString() { 
		return "B<" + this.X + ">";  
	}
}

/* 5. In a separate class named Part1, define the static main() method. In that same class, 
define a polymorphic static method, insertIntoMyList(), which takes two parameters, z and L, 
where L can be any MyList<> that z can be inserted into. The method insertIntoMyList() should 
insert z onto the end of L. Note that if z is a T object, for some type T, it is not the case 
that L must be a MyList<T>. */
public class Part1 {
	public static void main(String[] args){
		test();
	}
	static <T> void insertIntoMyList(T z, MyList<? super T> L){
		L.add(z);
	}
    public static void test() {
		MyList<A> m1 = new MyList<A>(new A(0));
		MyList<A> m2 = new MyList<A>(new A(0));
		for(Integer i = 1; i <= 5; i++) {
		    insertIntoMyList(new A(i),m1);
		    insertIntoMyList(new B(i),m2);
		}
	
		insertIntoMyList(new A(6),m2);
		System.out.println("m1 = " + m1);
		System.out.println("m2 = " + m2);
		int result = m1.compareTo(m2);
		String s = (result < 0) ? "less than" : (result == 0) ? "equal to" : "greater than";
		System.out.println("m1 is " + s + " m2");
		System.out.println("m1 + m2 = " + m1.plus(m2));
    }
}

/* 6. Finally, in class Part1, put the following method definition (test function) */
\end{minted}

\end{document}