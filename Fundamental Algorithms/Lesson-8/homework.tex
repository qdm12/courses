\documentclass[11pt]{article}
\pagestyle{empty}
\usepackage{color}
\usepackage{listings}
\usepackage{minted}
\usepackage{fancyhdr}
\usepackage{lastpage}
\usepackage[options]{algorithm2e}
\pagestyle{fancy}
\renewcommand{\headrulewidth}{0pt}
\cfoot[R]{\thepage~of~\pageref{LastPage}}
\addtolength{\oddsidemargin}{-.875in}
\addtolength{\evensidemargin}{-.875in}
\addtolength{\textwidth}{1.75in}
\addtolength{\topmargin}{-.875in}
\addtolength{\textheight}{1.75in}

\begin{document}
\begin{center} {\Large\bf FA, Homework 8} \\ Quentin McGaw (qm301) \\ 04/06/17
\end{center}

\begin{quote}
The search for truth is more precious than its possession
\\ --  Albert Einstein
\end{quote}

\begin{enumerate}
\item \textbf{\textcolor{blue}{Set $W=\lfloor{\sqrt{N}}\rfloor$.  
We are {\em given} $PRICE[I]$, $1\leq I\leq W$, the price of a rod of length $I$. 
Give a program that will output the optimal revenue for a rod of length $N^2$ 
and give the time, in $\Theta$-land,  of the algorithm.
Use an auxilliary array $R[J]$, $0\leq J\leq N^2$, where $R[J]$ will give the
optimal revenue for a rod of length $J$.  You may {\bf not} use
the term {\tt MAX} nor {\tt MIN} in your program  Explain, in clear words, how your program is working.
(You can and should use {\tt MAX} in your explanations.)}}
    \\ We have prices for rods of length up to $W=\lfloor{\sqrt{N}}\rfloor$ hence we need each cut piece to be of 
    length $W$ at most. Given that we have a rod of length $N^2$, we first cut it into $N^{3/2}$ different pieces.
    We then find the optimal revenue for each piece of length $W$ with the following algorithm:
    \begin{algorithm}[H]
        R = [0..W] \\
        R[0] = 0 \\
        \For{i from 1 to W}{
            q = PRICE[i] \\
            \For{j from 1 to i - 1}{
                \If{PRICE[j] + R[i - j] $>$ q}{
                    q = PRICE[j] + R[i - j]
                }
            }
            R[j] = q
        }
        \Return{R[W]}
    \end{algorithm}
    \\ The optimal revenue is then $N^{3/2}$ times the result returned from this algorithm.
    \\ Algorithm takes $\Theta(n^2)$.
\item  \textbf{\textcolor{blue}{(*) {\bf DO NOT SUBMIT -- NOT YET COVERED}
Suppose that the Huffman Code for $\{v,w,x,y,z\}$ has $0$ or $1$ as the
code word for $z$.  {\em Prove} that the frequency for $z$ cannot be less than
$\frac{1}{3}$.  {\em Give an example} where the frequency for $z$ is $0.36$ and
$z$ does get code word $0$ or $1$.}}
\item  \textbf{\textcolor{blue}{Suppose, in the Activity Selector problem, we
instead select the last activity to start that is compatible with all previously
selected activities. Describe how this
approach works, write a program for it (psuedocode allowed) and prove that it
yields an optimal algorithm.}}
    \\ Let $S$ be the set of activities and $A$ be the maximum-size subset of mutually compatible activities of $S$. We first arrange the activities in order of decreasing start time. Let $a_k = [s_k, f_k)$ be the last activity in $A$. Let $a_m$ be the activity with the earliest finish time. At every step, we discard any overlapping activities with already chosen activities. The latest starting activities remains chosen.
    \\\\ If $a_k = a_m$ then we found an optimal solution
    \\ Else, we build the new set starting with $a_k$ and go up until $a_m \leq a_k$. This set has the same number of activities and is mutually compatible with the original set. The optimal solution for $S$ maps directly to the optimal solution in the new set.    

\item \textbf{\textcolor{blue}{Students (professors too!) often come up with very clever ideas for
optimization programs.  The problem (often!) is that they (sometimes, but that
is enough) give the wrong answer.  Here are three approaches and {\em your} problem,
in each case, is to give an example where it yields the wrong answer.}}
    \begin{enumerate}
    \item \textbf{\textcolor{blue}{Pick the activity of the shortest duration 
    from amongst those which do not overlap previously selected activities.}}
        \\ With the set of activities $A = \{a_1, a_2, a_3\}$ where their start times are $s = \{2, 0, 3\}$ and finish times are $f = \{4, 3, 6\}$. Then their respective durations are $d = \{2, 3, 3\}$.
        \\ The solution obtained would be $\{a_1\}$ whilst the optimal solution is actually $\{a_2, a_3\}$.
    \item \textbf{\textcolor{blue}{(*) Pick the activity which overlaps the fewest other remaining activities
    from amongst those which do not overlap previously selected activities.}}
        \\ With the set of activities $A = \{a_1, a_2, a_3, ..., a_9\}$ where their start times are $s = \{0, 1, 1, 2, 3, 4, 5, 5, 6\}$ and finish times are $f = \{2, 3, 3, 4, 5, 6, 7, 7, 8\}$. Then their respective number of overlaps are $L = \{2, 3, 3, 3, 2, 3, 3, 3, 2\}$
        \\ The solution obtained would be $\{a_1, a_5, a_9\}$ whilst the optimal solution is actually $\{a_1, a_9, a_4, a_6\}$.
    \item \textbf{\textcolor{blue}{Pick the activity with the earliest start time
    from amongst those which do not overlap previously selected activities.}}
        \\ With the set of activities $A = \{a_1, a_2, a_3\}$ where their start times are $s = \{0, 1, 3\}$ and finish times are $f = \{4, 2, 4\}$.
        \\ The solution obtained would be $\{a_1\}$ whilst the optimal solution is actually $\{a_2, a_3\}$.
    \end{enumerate}
\item \textbf{\textcolor{blue}{{\bf DO NOT SUBMIT -- NOT YET COVERED}}}
    \begin{enumerate}
    \item \textbf{\textcolor{blue}{What is an optimal Huffman code for the following code when the
    frequencies are the first eight Fibonacci number?
    \[ a:1, b:1, c:2, d:3, e:5, f:8, g:13, h:21  \]}}
    \item \textbf{\textcolor{blue}{The Fibonacci sequence is defined by initial values $0,1$ 
    with each further term the sum of the previous two terms.  Generalize
    the previous answer to find the optimal code when there are $n$ letters with
    frequencies the first $n$ (excluding the $0$) Fibonacci numbers.}}
    \end{enumerate}
\item \textbf{\textcolor{blue}{{\bf DO NOT SUBMIT -- NOT YET COVERED} 
Suppose that in implementing the Huffman code we weren't
so clever as to use Min-Heaps.  Rather, at each step we found
the two letters of minimal frequency and replaced them by
a new letter with frequency their sum.  How long would that
algorithm take, in Thetaland, as a function of the initial number of
letters $n$.}}
\end{enumerate}


\begin{quote}
Dear Sir,
\\  I beg to introduce myself to you as a clerk in the Accounts Department
of the Port Trust Office at Madras on a salary of only \pounds 20 per
annum.  I am now about 25 years of age.  I have no University education but
I have undergone the ordinary school course.  After leaving school I have
been employing the spare time at my disposal to work at Mathematics\ldots
I am striking out a new path for myself.  I have made a special investigation
of divergent series in general and the results I get are termed
by the local mathematicians as ``startling"  \\ -- Ramanujan

\end{quote}

\begin{quote}
The truth is that the theory of primes is full of pitfalls, to surmount which requires the 
fullest of trainings in modern rigorous methods. This you are naturally without. I hope 
you will not be discouraged by my criticisms. I think your argument a very remarkable 
and ingenious one. To have proved what you claimed to have proved would have been
 about the most remarkable mathematical feat in the whole history of mathematics.
\\ G.H. Hardy, letter to Ramanujan, 1913 [after finding error in Ramanujan argument]
\end{quote}

\end{document}

