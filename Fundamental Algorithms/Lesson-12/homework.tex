\documentclass[11pt]{article}
\pagestyle{empty}
\usepackage{color}
\usepackage{fancyhdr}
\usepackage{lastpage}
\pagestyle{fancy}
\renewcommand{\headrulewidth}{0pt}
\cfoot[R]{\thepage~of~\pageref{LastPage}}
\addtolength{\oddsidemargin}{-.875in}
\addtolength{\evensidemargin}{-.875in}
\addtolength{\textwidth}{1.75in}
\addtolength{\topmargin}{-.875in}
\addtolength{\textheight}{1.75in}

\begin{document}
\begin{center} {\Large\bf FA, Homework 12} \\ Quentin McGaw (qm301) \\ 05/04/17
\end{center}

\begin{quote}
To me, it does not seem unlikely that on some shelf of the uni-
verse there lies a total book. I pray the unknown gods that some
man - even if only one man, and though it have been thousands
of years ago! - may have examined and read it. If honor and wis-
dom and happiness are not for me, let them be for others. May
heaven exist, though my place be in hell. Let me be outraged
and annihilated, but may Thy enormous Library be justified, for
one instant, in one being.
\\ from The Library of Babel by Jorge Luis Borges
\end{quote}

\begin{center}
    \begin{tabular}{|c | c | c |} 
    \hline
    Problem Type & Verifiable in P time & Solvable in P time \\
    \hline
    P & YES & YES \\
    \hline
    NP & YES & YES OR NO \\
    \hline
    NP-complete & YES & UNKNOWN \\
    \hline
    NP-hard & YES OR NO & UNKNOWN \\
    \hline
    \end{tabular}
\end{center}
\textbf{THEOREM:} $\forall L \in NP \mbox{, } L \leq 3-SAT$
\\ \textbf{COROLLARY: } $3-SAT \in P \Rightarrow P = NP$
\\ Note: The SAT problem is given the expression, is there some assignment of TRUE and FALSE values to the variables that will make the entire expression true. SAT3 problem is a special case of SAT problem, where Boolean expression must be divided into clauses of containing three literals.

\begin{enumerate}
\item \textbf{\textcolor{blue}{Which of the following problem classes are in $P$ and which are probably not in $P$.  (By probably not we mean that we do not as of today know that it is in $P$ but of course tomorrow somebody might come up with a clever algorithm.))}}
\begin{enumerate}
    \item \textbf{\textcolor{blue}{{\tt PRIME}. The input here would be the integer $n$ and Yes would be returned iff $n$ is prime.}}
        \\ With the AKS algorithm (Agarwal, Kayal, Sexana), PRIME can be found in polynomial time (today in $O((\log(n))^6)$ time) and is therefore in P.
    \item \textbf{\textcolor{blue}{I gave the above problem twenty years ago.  What was the answer then?}}
        \\ The AKS algorithm was found in 2002 and before that PRIME could not be found in polynomial time so it was not in P.
    \item \textbf{\textcolor{blue}{{\tt CONNECTED-GRAPH}.  The input here would be a graph $G$ and Yes would be returned iff the graph was connected.}}
        \\ We can use BFS for this problem which takes $O(|V| + |E|)$ (polynomial time) so CONNECTED-GRAPH is in P.
    \item \textbf{\textcolor{blue}{{\tt TRAVELING-SALESMAN}. The input here would be a graph $G$ together with a positive integer weight $w(e)$ for each edge $e$ and an integer $B$.  Yes would be returned iff there was a Hamiltonian Cycle which had total weight at most $B$.}}
        \\ \textbf{TSP: } "Given a list of cities and the distances between each pair of cities, what is the shortest possible route that visits each city exactly once and returns to the origin city?"
        \\ \textbf{Hamiltonian cycle: } Path in a graph that visits each vertex exactly once.
        \\ This is a problem class in NP-hard and probably not in P.
    \item \textbf{\textcolor{blue}{{\tt SPANNING-TREE}. The input here would be a graph $G$ together with a positive integer weight $w(e)$ for each edge $e$ and an integer $B$.  Yes would be returned iff there was a spanning tree which had total weight at most $B$.}}
        \\ Kruskal's algorithm (requires $O(E\log(V))$ time) or Prim's algorithm (requires $O(E\log(V))$ time with adjacency lists) can be used to solve this problem. Therefore this problem class is in P.
    \item \textbf{\textcolor{blue}{{\tt ALMOSTDAG}. The input here would be a directed graph $G$. Yes would be returned iff there was a set of at most $10$ edges of $G$ that could be removed from $G$ so that the remaining graph is a {\tt DAG}. (Your argument should work with $10$ replaced by any {\em constant} value.)}}
        \begin{itemize}
            \item For a \textbf{directed graph} with n vertices, there are at most $n(n-1)$ edges.
            \item To test if a graph is a DAG, we can use TOP-SORT which requires $O(|V|+|E|) = O(n^2)$ \textcolor{red}{time}.
            \item There are a \textcolor{red}{total of} $(n^2)^{10} = n^{20}$ sets of 10 edges, which is also the number of graphs to test.
            \item So we need $O(n^{22})$ time overall, which is still polynomial
        \end{itemize}
\end{enumerate}
    
\item \textbf{\textcolor{blue}{Show that the following problem classes are in $NP$. (That is, describe the certificate that the Oracle gives and describe the procedure that Verifier will take. Warning: Do not trust Oracle! For example, if Oracle gives you $n$ distinct vertices you have to verify that they are indeed distinct!)}}
\begin{enumerate}
    \item \textbf{\textcolor{blue}{{\tt PRIME-INTERVAL} The input here would be integers $n,a,b$. Yes would be returned iff there was a prime $p$ which divided $n$ and for which $a\leq p\leq b$.}}
        \\ Oracle gives a prime $p$. The verifier checks that:
        \begin{itemize}
            \item $a \leq p \leq b$
            \item $p | n$
            \item p is prime, using the AKS algorithm
        \end{itemize}
    \item \textbf{\textcolor{blue}{{\tt TRAVELING-SALESMAN} As described above.}}
        \\ Oracle gives the ordering $x_1, ..., x_n$ of vertices. The verifier checks that:
        \begin{itemize}
            \item The vertices are distinct
            \item The vertices are all the vertices
            \item The sum of the weights of the edges $\{x_i, x_{i+1}\}$ (including $\{x_n, x_1\}$) is at most B.
        \end{itemize}
    \item \textbf{\textcolor{blue}{{\tt COMPOSITE} The input here would be an integer $n$. Yes would be returned if $n$ was composite.}}
    \begin{enumerate}
        \item \textbf{\textcolor{blue}{Give a solution that uses the Agarwal, Kayal, Saxena algorithm.}}
            \\ No need for the oracle, the verifier can directly use AKS and then flip the YES/NO answer.
            \\ This is because any $L$ which is in $P$ is in $NP$ since the oracle is not used.
        \item \textbf{\textcolor{blue}{Give a solution not using the AKS algorithm.}}
            \\ The oracle gives $a$, $b$ with $n = ab$. The verifier checks the multiplication.
        \end{enumerate}
    \item \textbf{\textcolor{blue}{{\tt 3-COLOR}. The input here would be a graph $G$. Yes would be returned if there was a three coloring of the vertices such that no two adjacent vertices $v,w$ had the same color.}}
        \\ Oracle gives Yes and the 3 coloring. The verifier checks that for every vertex v and w $\in$ Adj[v], v and w do not have the same color.
    \item \textbf{\textcolor{blue}{{\tt NEAR-DAG}. The input here would be a directed graph $G$ and an integer $B$. Yes would be returned if there was a set of at most $B$ edges that could be removed from $G$ so that the remaining graph was acyclic. (This is like {\tt ALMOST-DAG} with the critical distinction that $B$ is not restricted to $10$, or any constant value. Rather, $B$ can depend on the number of vertices of $G$.)}}
        \\ Oracle gives the B edges to be removed. The verifier checks that:
        \begin{itemize}
            \item The B edges are in the graph
            \item Produce G' by removing the B edges from G, and applies TOP-SORT to it to check if it is a DAG.
        \end{itemize}
\end{enumerate}
    
\item \textbf{\textcolor{blue}{For the following pairs $L_1,L_2$ of problem classes show that $L_1\leq_P L_2$.  That is, given a ``black box" that will solve any instance of $L_2$ in unit time, create a polynomial time algorithm that will solve any instance of $L_1$ in polynomial time.}}
\begin{enumerate}
    \item \textbf{\textcolor{blue}{Let $L_2$ be {\tt TRAVELLING-SALESMAN DESIGNATED PATH}. The input here would be a graph $G$, two designated vertices, a source $v_1$ and a sink $v_n$, together with a positive integer weight $w(e)$ for each edge $e$ and an integer $B$.  Yes would be returned iff there was a Hamiltonian Path (i.e., one goes through all the vertices $v_1,\ldots,v_n$ in some order (starting and ending at the designated vertices) but does {\em not} return from $v_n$ back to $v_1$) which had total weight at most $B$. $L_1$ is {\tt TRAVELLING-SALESMAN} as described above.}}
        \\ For each edge $e = \{x, y\}$, ask $L_2$ if there is a Hamiltonian path from x to y whose length is at most $B - w(e)$.
        \begin{itemize}
            \item The answer is \textbf{YES}: $L_1$ is YES, as you add the edge $e$ to the Hamiltonian path.
            \item The answer is always \textbf{NO}: $L_1$ is NO, as a Hamiltonian cycle of length $\leq$ B would have to use some $e = \{x,y\}$, and removing it would give a path of length $< B - w(e)$ with that source and sink.
        \end{itemize}
    \item \textbf{\textcolor{blue}{Let $L_2$ be {\tt CLIQUE}. The input here would be a graph $G$ together with a positive integer $B$. Yes would be returned iff there was a clique with at least $B$ vertices. (A set of vertices in a graph $G$ is a clique if {\em every} pair of them are adjacent.) Let $L_1$ be {\tt INDEPENDENT-SET}. The input here would be a graph $G$ together with a positive integer $B$. Yes would be returned iff there was a independent set with at least $B$ vertices. (A set of vertices in a graph $G$ is an independent set if {\em no} pair of them are adjacent.)}}
        \\ G has an independent set, with size $\geq B \Leftrightarrow G^c$ has a clique of size $\geq B$ ($G^c$ is the complement of $G$).
        \\ Pairs of vertices are adjacent in $G^c \Leftrightarrow$ they are not adjacent in $G$.
        \\ To create $G^c$ from $G$, we need $O(n^2)$.
        \\ The algorithm for $L_1$ would be to create $G^c$ and then apply $L_2$ to it.
\end{enumerate}

\item \textbf{\textcolor{blue}{Assume {\tt PRIME-INTERVAL} (defined above) is in $P$. Using it as a black box give a polynomial time algorithm with input integer $n\geq 2$ that returns some prime factor $p$.  (Caution: This means polynomial in the number of digits in $n$, or what is sometimes called polylog $n$, meaning $O(\lg^cn)$ for some constant $c$.)}}
    \begin{itemize}
        \item PRIME-INTERVAL(n,2,n) is TRUE
        \item We check PRIME-INTERVAL(n,2,$\frac{n}{2}$)
        \begin{itemize}
            \item TRUE: there is a prime in $[2, \frac{n}{2}]$.
            \item FALSE: there is a prime in $(\frac{n}{2}, n]$.
        \end{itemize}
        \item We keep on splitting the interval until we have an interval of length 1 where there is a prime.
    \end{itemize}
    This takes $\log(n)$ calls
\\ \textbf{\textcolor{blue}{(*) Further, give a polynomial time algorithm with input integer $n\geq 2$ that returns the entire prime factorization of $n$.}}
    \begin{itemize}
        \item Apply this algorithm to get the first prime factor $p$. 
        \item Iterate the procedure on $\frac{n}{p}$.
        \item This is done at most $\log(n)$ times because each time, the new value of $n$ is at most the halved value.
        \item Because each procedure requires $\log(n)$ time, we need $O(\log^2(n))$ time overall.
    \end{itemize}

\end{enumerate}

\end{document}